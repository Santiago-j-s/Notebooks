\documentclass{article}
    
    \usepackage{graphicx} % Used to insert images
    \usepackage{adjustbox} % Used to constrain images to a maximum size 
    \usepackage{color} % Allow colors to be defined
    \usepackage{enumerate} % Needed for markdown enumerations to work
    \usepackage{geometry} % Used to adjust the document margins
    \usepackage{amsmath} % Equations
    \usepackage{amssymb} % Equations
    \usepackage[mathletters]{ucs} % Extended unicode (utf-8) support
    \usepackage[utf8x]{inputenc} % Allow utf-8 characters in the tex document
    \usepackage{fancyvrb} % verbatim replacement that allows latex
    \usepackage{grffile} % extends the file name processing of package graphics 
                         % to support a larger range 
    % The hyperref package gives us a pdf with properly built
    % internal navigation ('pdf bookmarks' for the table of contents,
    % internal cross-reference links, web links for URLs, etc.)
    \usepackage{hyperref}
    \usepackage{longtable} % longtable support required by pandoc >1.10
    \usepackage{booktabs}  % table support for pandoc > 1.12.2
    
    \definecolor{orange}{cmyk}{0,0.4,0.8,0.2}
    \definecolor{darkorange}{rgb}{.71,0.21,0.01}
    \definecolor{darkgreen}{rgb}{.12,.54,.11}
    \definecolor{myteal}{rgb}{.26, .44, .56}
    \definecolor{gray}{gray}{0.45}
    \definecolor{lightgray}{gray}{.95}
    \definecolor{mediumgray}{gray}{.8}
    \definecolor{inputbackground}{rgb}{.95, .95, .85}
    \definecolor{outputbackground}{rgb}{.95, .95, .95}
    \definecolor{traceback}{rgb}{1, .95, .95}
    % ansi colors
    \definecolor{red}{rgb}{.6,0,0}
    \definecolor{green}{rgb}{0,.65,0}
    \definecolor{brown}{rgb}{0.6,0.6,0}
    \definecolor{blue}{rgb}{0,.145,.698}
    \definecolor{purple}{rgb}{.698,.145,.698}
    \definecolor{cyan}{rgb}{0,.698,.698}
    \definecolor{lightgray}{gray}{0.5}
    
    % bright ansi colors
    \definecolor{darkgray}{gray}{0.25}
    \definecolor{lightred}{rgb}{1.0,0.39,0.28}
    \definecolor{lightgreen}{rgb}{0.48,0.99,0.0}
    \definecolor{lightblue}{rgb}{0.53,0.81,0.92}
    \definecolor{lightpurple}{rgb}{0.87,0.63,0.87}
    \definecolor{lightcyan}{rgb}{0.5,1.0,0.83}
    
    % commands and environments needed by pandoc snippets
    % extracted from the output of `pandoc -s`
    \DefineVerbatimEnvironment{Highlighting}{Verbatim}{commandchars=\\\{\}}
    % Add ',fontsize=\small' for more characters per line
    \newenvironment{Shaded}{}{}
    \newcommand{\KeywordTok}[1]{\textcolor[rgb]{0.00,0.44,0.13}{\textbf{{#1}}}}
    \newcommand{\DataTypeTok}[1]{\textcolor[rgb]{0.56,0.13,0.00}{{#1}}}
    \newcommand{\DecValTok}[1]{\textcolor[rgb]{0.25,0.63,0.44}{{#1}}}
    \newcommand{\BaseNTok}[1]{\textcolor[rgb]{0.25,0.63,0.44}{{#1}}}
    \newcommand{\FloatTok}[1]{\textcolor[rgb]{0.25,0.63,0.44}{{#1}}}
    \newcommand{\CharTok}[1]{\textcolor[rgb]{0.25,0.44,0.63}{{#1}}}
    \newcommand{\StringTok}[1]{\textcolor[rgb]{0.25,0.44,0.63}{{#1}}}
    \newcommand{\CommentTok}[1]{\textcolor[rgb]{0.38,0.63,0.69}{\textit{{#1}}}}
    \newcommand{\OtherTok}[1]{\textcolor[rgb]{0.00,0.44,0.13}{{#1}}}
    \newcommand{\AlertTok}[1]{\textcolor[rgb]{1.00,0.00,0.00}{\textbf{{#1}}}}
    \newcommand{\FunctionTok}[1]{\textcolor[rgb]{0.02,0.16,0.49}{{#1}}}
    \newcommand{\RegionMarkerTok}[1]{{#1}}
    \newcommand{\ErrorTok}[1]{\textcolor[rgb]{1.00,0.00,0.00}{\textbf{{#1}}}}
    \newcommand{\NormalTok}[1]{{#1}}
    
    % Define a nice break command that doesn't care if a line doesn't already
    % exist.
    \def\br{\hspace*{\fill} \\* }
    % Math Jax compatability definitions
    \def\gt{>}
    \def\lt{<}
    % Document parameters
    \title{Pr?ctico 3}

    % Pygments definitions
    
\makeatletter
\def\PY@reset{\let\PY@it=\relax \let\PY@bf=\relax%
    \let\PY@ul=\relax \let\PY@tc=\relax%
    \let\PY@bc=\relax \let\PY@ff=\relax}
\def\PY@tok#1{\csname PY@tok@#1\endcsname}
\def\PY@toks#1+{\ifx\relax#1\empty\else%
    \PY@tok{#1}\expandafter\PY@toks\fi}
\def\PY@do#1{\PY@bc{\PY@tc{\PY@ul{%
    \PY@it{\PY@bf{\PY@ff{#1}}}}}}}
\def\PY#1#2{\PY@reset\PY@toks#1+\relax+\PY@do{#2}}

\expandafter\def\csname PY@tok@cp\endcsname{\def\PY@tc##1{\textcolor[rgb]{0.74,0.48,0.00}{##1}}}
\expandafter\def\csname PY@tok@gr\endcsname{\def\PY@tc##1{\textcolor[rgb]{1.00,0.00,0.00}{##1}}}
\expandafter\def\csname PY@tok@err\endcsname{\def\PY@bc##1{\setlength{\fboxsep}{0pt}\fcolorbox[rgb]{1.00,0.00,0.00}{1,1,1}{\strut ##1}}}
\expandafter\def\csname PY@tok@il\endcsname{\def\PY@tc##1{\textcolor[rgb]{0.40,0.40,0.40}{##1}}}
\expandafter\def\csname PY@tok@sd\endcsname{\let\PY@it=\textit\def\PY@tc##1{\textcolor[rgb]{0.73,0.13,0.13}{##1}}}
\expandafter\def\csname PY@tok@w\endcsname{\def\PY@tc##1{\textcolor[rgb]{0.73,0.73,0.73}{##1}}}
\expandafter\def\csname PY@tok@kn\endcsname{\let\PY@bf=\textbf\def\PY@tc##1{\textcolor[rgb]{0.00,0.50,0.00}{##1}}}
\expandafter\def\csname PY@tok@gd\endcsname{\def\PY@tc##1{\textcolor[rgb]{0.63,0.00,0.00}{##1}}}
\expandafter\def\csname PY@tok@vi\endcsname{\def\PY@tc##1{\textcolor[rgb]{0.10,0.09,0.49}{##1}}}
\expandafter\def\csname PY@tok@ow\endcsname{\let\PY@bf=\textbf\def\PY@tc##1{\textcolor[rgb]{0.67,0.13,1.00}{##1}}}
\expandafter\def\csname PY@tok@o\endcsname{\def\PY@tc##1{\textcolor[rgb]{0.40,0.40,0.40}{##1}}}
\expandafter\def\csname PY@tok@gp\endcsname{\let\PY@bf=\textbf\def\PY@tc##1{\textcolor[rgb]{0.00,0.00,0.50}{##1}}}
\expandafter\def\csname PY@tok@ne\endcsname{\let\PY@bf=\textbf\def\PY@tc##1{\textcolor[rgb]{0.82,0.25,0.23}{##1}}}
\expandafter\def\csname PY@tok@mh\endcsname{\def\PY@tc##1{\textcolor[rgb]{0.40,0.40,0.40}{##1}}}
\expandafter\def\csname PY@tok@nv\endcsname{\def\PY@tc##1{\textcolor[rgb]{0.10,0.09,0.49}{##1}}}
\expandafter\def\csname PY@tok@c\endcsname{\let\PY@it=\textit\def\PY@tc##1{\textcolor[rgb]{0.25,0.50,0.50}{##1}}}
\expandafter\def\csname PY@tok@nf\endcsname{\def\PY@tc##1{\textcolor[rgb]{0.00,0.00,1.00}{##1}}}
\expandafter\def\csname PY@tok@kc\endcsname{\let\PY@bf=\textbf\def\PY@tc##1{\textcolor[rgb]{0.00,0.50,0.00}{##1}}}
\expandafter\def\csname PY@tok@mf\endcsname{\def\PY@tc##1{\textcolor[rgb]{0.40,0.40,0.40}{##1}}}
\expandafter\def\csname PY@tok@sr\endcsname{\def\PY@tc##1{\textcolor[rgb]{0.73,0.40,0.53}{##1}}}
\expandafter\def\csname PY@tok@bp\endcsname{\def\PY@tc##1{\textcolor[rgb]{0.00,0.50,0.00}{##1}}}
\expandafter\def\csname PY@tok@k\endcsname{\let\PY@bf=\textbf\def\PY@tc##1{\textcolor[rgb]{0.00,0.50,0.00}{##1}}}
\expandafter\def\csname PY@tok@m\endcsname{\def\PY@tc##1{\textcolor[rgb]{0.40,0.40,0.40}{##1}}}
\expandafter\def\csname PY@tok@gi\endcsname{\def\PY@tc##1{\textcolor[rgb]{0.00,0.63,0.00}{##1}}}
\expandafter\def\csname PY@tok@go\endcsname{\def\PY@tc##1{\textcolor[rgb]{0.53,0.53,0.53}{##1}}}
\expandafter\def\csname PY@tok@kr\endcsname{\let\PY@bf=\textbf\def\PY@tc##1{\textcolor[rgb]{0.00,0.50,0.00}{##1}}}
\expandafter\def\csname PY@tok@nt\endcsname{\let\PY@bf=\textbf\def\PY@tc##1{\textcolor[rgb]{0.00,0.50,0.00}{##1}}}
\expandafter\def\csname PY@tok@cs\endcsname{\let\PY@it=\textit\def\PY@tc##1{\textcolor[rgb]{0.25,0.50,0.50}{##1}}}
\expandafter\def\csname PY@tok@nc\endcsname{\let\PY@bf=\textbf\def\PY@tc##1{\textcolor[rgb]{0.00,0.00,1.00}{##1}}}
\expandafter\def\csname PY@tok@gu\endcsname{\let\PY@bf=\textbf\def\PY@tc##1{\textcolor[rgb]{0.50,0.00,0.50}{##1}}}
\expandafter\def\csname PY@tok@kp\endcsname{\def\PY@tc##1{\textcolor[rgb]{0.00,0.50,0.00}{##1}}}
\expandafter\def\csname PY@tok@ge\endcsname{\let\PY@it=\textit}
\expandafter\def\csname PY@tok@nn\endcsname{\let\PY@bf=\textbf\def\PY@tc##1{\textcolor[rgb]{0.00,0.00,1.00}{##1}}}
\expandafter\def\csname PY@tok@gs\endcsname{\let\PY@bf=\textbf}
\expandafter\def\csname PY@tok@na\endcsname{\def\PY@tc##1{\textcolor[rgb]{0.49,0.56,0.16}{##1}}}
\expandafter\def\csname PY@tok@cm\endcsname{\let\PY@it=\textit\def\PY@tc##1{\textcolor[rgb]{0.25,0.50,0.50}{##1}}}
\expandafter\def\csname PY@tok@sx\endcsname{\def\PY@tc##1{\textcolor[rgb]{0.00,0.50,0.00}{##1}}}
\expandafter\def\csname PY@tok@no\endcsname{\def\PY@tc##1{\textcolor[rgb]{0.53,0.00,0.00}{##1}}}
\expandafter\def\csname PY@tok@se\endcsname{\let\PY@bf=\textbf\def\PY@tc##1{\textcolor[rgb]{0.73,0.40,0.13}{##1}}}
\expandafter\def\csname PY@tok@kd\endcsname{\let\PY@bf=\textbf\def\PY@tc##1{\textcolor[rgb]{0.00,0.50,0.00}{##1}}}
\expandafter\def\csname PY@tok@mo\endcsname{\def\PY@tc##1{\textcolor[rgb]{0.40,0.40,0.40}{##1}}}
\expandafter\def\csname PY@tok@gt\endcsname{\def\PY@tc##1{\textcolor[rgb]{0.00,0.27,0.87}{##1}}}
\expandafter\def\csname PY@tok@mi\endcsname{\def\PY@tc##1{\textcolor[rgb]{0.40,0.40,0.40}{##1}}}
\expandafter\def\csname PY@tok@sc\endcsname{\def\PY@tc##1{\textcolor[rgb]{0.73,0.13,0.13}{##1}}}
\expandafter\def\csname PY@tok@si\endcsname{\let\PY@bf=\textbf\def\PY@tc##1{\textcolor[rgb]{0.73,0.40,0.53}{##1}}}
\expandafter\def\csname PY@tok@gh\endcsname{\let\PY@bf=\textbf\def\PY@tc##1{\textcolor[rgb]{0.00,0.00,0.50}{##1}}}
\expandafter\def\csname PY@tok@c1\endcsname{\let\PY@it=\textit\def\PY@tc##1{\textcolor[rgb]{0.25,0.50,0.50}{##1}}}
\expandafter\def\csname PY@tok@ni\endcsname{\let\PY@bf=\textbf\def\PY@tc##1{\textcolor[rgb]{0.60,0.60,0.60}{##1}}}
\expandafter\def\csname PY@tok@sh\endcsname{\def\PY@tc##1{\textcolor[rgb]{0.73,0.13,0.13}{##1}}}
\expandafter\def\csname PY@tok@nb\endcsname{\def\PY@tc##1{\textcolor[rgb]{0.00,0.50,0.00}{##1}}}
\expandafter\def\csname PY@tok@s\endcsname{\def\PY@tc##1{\textcolor[rgb]{0.73,0.13,0.13}{##1}}}
\expandafter\def\csname PY@tok@vg\endcsname{\def\PY@tc##1{\textcolor[rgb]{0.10,0.09,0.49}{##1}}}
\expandafter\def\csname PY@tok@s2\endcsname{\def\PY@tc##1{\textcolor[rgb]{0.73,0.13,0.13}{##1}}}
\expandafter\def\csname PY@tok@ss\endcsname{\def\PY@tc##1{\textcolor[rgb]{0.10,0.09,0.49}{##1}}}
\expandafter\def\csname PY@tok@kt\endcsname{\def\PY@tc##1{\textcolor[rgb]{0.69,0.00,0.25}{##1}}}
\expandafter\def\csname PY@tok@nd\endcsname{\def\PY@tc##1{\textcolor[rgb]{0.67,0.13,1.00}{##1}}}
\expandafter\def\csname PY@tok@sb\endcsname{\def\PY@tc##1{\textcolor[rgb]{0.73,0.13,0.13}{##1}}}
\expandafter\def\csname PY@tok@vc\endcsname{\def\PY@tc##1{\textcolor[rgb]{0.10,0.09,0.49}{##1}}}
\expandafter\def\csname PY@tok@s1\endcsname{\def\PY@tc##1{\textcolor[rgb]{0.73,0.13,0.13}{##1}}}
\expandafter\def\csname PY@tok@nl\endcsname{\def\PY@tc##1{\textcolor[rgb]{0.63,0.63,0.00}{##1}}}
\expandafter\def\csname PY@tok@mb\endcsname{\def\PY@tc##1{\textcolor[rgb]{0.40,0.40,0.40}{##1}}}

\def\PYZbs{\char`\\}
\def\PYZus{\char`\_}
\def\PYZob{\char`\{}
\def\PYZcb{\char`\}}
\def\PYZca{\char`\^}
\def\PYZam{\char`\&}
\def\PYZlt{\char`\<}
\def\PYZgt{\char`\>}
\def\PYZsh{\char`\#}
\def\PYZpc{\char`\%}
\def\PYZdl{\char`\$}
\def\PYZhy{\char`\-}
\def\PYZsq{\char`\'}
\def\PYZdq{\char`\"}
\def\PYZti{\char`\~}
% for compatibility with earlier versions
\def\PYZat{@}
\def\PYZlb{[}
\def\PYZrb{]}
\makeatother

    % Exact colors from NB
    \definecolor{incolor}{rgb}{0.0, 0.0, 0.5}
    \definecolor{outcolor}{rgb}{0.545, 0.0, 0.0}
    
    % Prevent overflowing lines due to hard-to-break entities
    \sloppy 
    % Setup hyperref package
    \hypersetup{
      breaklinks=true,  % so long urls are correctly broken across lines
      colorlinks=true,
      urlcolor=blue,
      linkcolor=darkorange,
      citecolor=darkgreen,
      }
    % Slightly bigger margins than the latex defaults
    
    \geometry{verbose,tmargin=1in,bmargin=1in,lmargin=1in,rmargin=1in}

    \begin{document}
    
    
    \maketitle
    
    \subsubsection{1. Enunciar las funciones de la Capa de Enlace de
Datos}\label{enunciar-las-funciones-de-la-capa-de-enlace-de-datos}

\begin{itemize}
\itemsep1pt\parskip0pt\parsep0pt
\item
  Enmarcado de tramas
\item
  Detección y corrección de errores de transmisión
\item
  Control de flujo
\item
  Sincronización de octeto y de carácter
\end{itemize}

\subsubsection{2. ¿Qué alternativa conoce para el Enmarcado de Tramas de
la Capa de
Enlace?}\label{quuxe9-alternativa-conoce-para-el-enmarcado-de-tramas-de-la-capa-de-enlace}

\begin{itemize}
\itemsep1pt\parskip0pt\parsep0pt
\item
  Conteo de carácteres
\item
  Relleno de bits
\item
  Relleno de carácteres
\item
  Violaciones de codificación de la capa física (Ej. Codificación
  Manchester)
\end{itemize}

\subsubsection{3. Si la cadena de bits 0111101111101111110 se rellena
¿Cuál es la cadena de
salida?}\label{si-la-cadena-de-bits-0111101111101111110-se-rellena-cuuxe1l-es-la-cadena-de-salida}

011110111110011111010

\subsubsection{4. Para una transmisión de códigos BCD de 4 bits, se
decide emplear un esquema de corrección de errores de un bit, mediante
Código
Hamming}\label{para-una-transmisiuxf3n-de-cuxf3digos-bcd-de-4-bits-se-decide-emplear-un-esquema-de-correcciuxf3n-de-errores-de-un-bit-mediante-cuxf3digo-hamming}

\begin{itemize}
\itemsep1pt\parskip0pt\parsep0pt
\item
  Plantear la tabla del código resultante a transmitir, indicando la
  lógica de los bits de paridad resultante.
\end{itemize}

\begin{longtable}[c]{@{}lll@{}}
\toprule
Decimal & Binario & Hamming\tabularnewline
\midrule
\endhead
0 & 0000 & 0000000\tabularnewline
1 & 0001 & 1101001\tabularnewline
2 & 0010 & 0101010\tabularnewline
3 & 0011 & 1000011\tabularnewline
4 & 0100 & 1001100\tabularnewline
5 & 0101 & 0100101\tabularnewline
6 & 0110 & 1100110\tabularnewline
7 & 0111 & 0001111\tabularnewline
8 & 1000 & 1110000\tabularnewline
9 & 1001 & 0011001\tabularnewline
\bottomrule
\end{longtable}

\begin{verbatim}
bit 1 = b3 ⊕ b5 ⊕ b7
bit 2 = b3 ⊕ b6 ⊕ b7
bit 4 = b5 ⊕ b6 ⊕ b7
\end{verbatim}

\begin{itemize}
\itemsep1pt\parskip0pt\parsep0pt
\item
  Suponer un error en la tranmisión del bit 5 del código BCD 9 (1001)
  ¿Cómo la corregiría el receptor?
\end{itemize}

Msg. Recibido = 0011101

Cálculo Hamming

\begin{verbatim}
bit 1 = 1
bit 2 = 0
bit 3 = 0
\end{verbatim}

OR entre los bits 1, 2 y 4 del mensaje recibido y los calculados según
Hamming:

100 ⊕ 001 = 101

Invierto -\textgreater{} 101 = 5

Hay un error en el bit 5, entonces el receptor sabe que tiene que
cambiar el uno en el bit 5 por un cero.

\subsubsection{5. Se desea enviar la clave 9815, en formato BCD,
asegurandolá mediante Código
Hamming.}\label{se-desea-enviar-la-clave-9815-en-formato-bcd-asegurandoluxe1-mediante-cuxf3digo-hamming.}

\begin{itemize}
\itemsep1pt\parskip0pt\parsep0pt
\item
  Codificar el mensaje a enviar
\end{itemize}

\begin{verbatim}
9 | 0011001
8 | 1110000
1 | 1101001
5 | 0100101
\end{verbatim}

\begin{itemize}
\itemsep1pt\parskip0pt\parsep0pt
\item
  El receptor al extraer la información recibida, ha obtenido la
  siguiente cadena de bits 0100001 para el último dígito. Efectuar la
  operación de comprobación.
\end{itemize}

Hamming

\begin{verbatim}
bit 1 = 1
bit 2 = 1
bit 3 = 1
\end{verbatim}

111 ⊕ 010 = 101

Invierto -\textgreater{} 101 - Error en el bit 5

Msg. Original 0100101

Extraigo bits de comprobación (1, 2 y 4): 0101 = 5

El último dígito del mensaje transmitido es 5

\subsubsection{6. En qué consiste el esquema CRC (Código de Redundancia
Cíclica) para detección de
errores}\label{en-quuxe9-consiste-el-esquema-crc-cuxf3digo-de-redundancia-cuxedclica-para-detecciuxf3n-de-errores}

El CRC es un valor usado para comprobar que los datos no se alteren
durante la transmisión. El transmisor cálcula un CRC y envía el
resultado junto con los datos. El receptor cálcula el CRC de los datos
recibidos y los compara con el CRC del paquete, si no coinciden el
paquete tuvo errores en su tranmisión. CRC no puede detectar todos los
errores posibles de tranmisión, por lo que es importante elegir un
esquema CRC que tenga una buena probabilidad de detección de errores.

\subsubsection{7. Utilizar el polinomio generador de la UIT para generar
el CRC correspondiente a un 1 seguido de 15
ceros.}\label{utilizar-el-polinomio-generador-de-la-uit-para-generar-el-crc-correspondiente-a-un-1-seguido-de-15-ceros.}

\$ \text{Polinomio generador de la UIT: } x\^{}\{16\} + x\^{}\{12\} +
x\^{}\{5\} + 1\$

\begin{align}
G(x)& =        10001000000100001 \\
M(x)& =        1000000000000000 \\
R(x)& =        0000000000000000 \\
M(x) + x^r& = 10000000000000000000000000000000 \\
\end{align}

\begin{verbatim}
10000000000000000000000000000000 | 10001000000100001
 0001000000100001                |
 ----------------                |
    10000001000010000            |
     0001000000100001            |
     ----------------            |
        1001000110001            |
         0001000000100001        |
         ----------------        |
           11001100110001000     |
            0001000000100001     |
            ----------------     |
            10001001101010010    |
             0001000000100001    |
             ----------------    |
                0001101110011000 |
\end{verbatim}

\$ CRC = 0001101110011000 \$

\subsubsection{8.}\label{section}

\begin{itemize}
\itemsep1pt\parskip0pt\parsep0pt
\item
  \textbf{Determinar el mensaje CRC a agregar al mensaje 1100011 con P =
  110011, analizar la capacidad de detección de G}
\end{itemize}

\begin{align}
G(x)& =        110011 \\
M(x)& =        1100011 \\
x^r& =         00000\\
M(x) + x^r& = 110001100000 \\
\end{align}

\begin{verbatim}
110001100000 | 110011
 10011       |
 -----       |
    100000   |
     10011   |
     -----   |
     100110  |
      10011  |
      -----  |
      101010 |
       10011 |
       ----- |
       11001 |
      
\end{verbatim}

\$ CRC = 11001 \$

\begin{itemize}
\itemsep1pt\parskip0pt\parsep0pt
\item
  \textbf{Indicar la secuencia de bits a transmitir luego de realizar
  ``bit stuffing''}
\end{itemize}

\$ M(x) + CRC = 110001111001 \$

La secuencia a transmitir sería 01111110 11000111001 01111110

\subsubsection{9.}\label{section-1}

\begin{itemize}
\itemsep1pt\parskip0pt\parsep0pt
\item
  \textbf{Determinar el CRC a agregar al mensaje 11011111 con P =
  101110}
\end{itemize}

\begin{align}
G(x)& =        101110 \\
M(x)& =        11011111 \\
x^r& =         00000 \\
M(x) + x^r& =  1101111100000 \\
\end{align}

\begin{verbatim}
1101111100000 | 101110
 01110        |
 -----        |
 110011       |
  01110       |
  -----       |
  111011      |
   01110      |
   -----      |
   101010     |
    01110     |
    -----     |
      100000  |
       01110  |
       -----  |
        11100 |
\end{verbatim}

\$ CRC = 11100 \$

\begin{itemize}
\itemsep1pt\parskip0pt\parsep0pt
\item
  \textbf{Indicar la secuencia de bits a transmitir luego de realizar
  ``bit stuffing''}
\end{itemize}

\$ M(x) + CRC = 1101111111100 \$

La secuencia a transmitir sería 01111110 11011111\textbf{0}11100
01111110

\subsubsection{10. ¿Cuál es la idea principal de los protocolos de
ventana corrediza en la capa de
enlace?}\label{cuuxe1l-es-la-idea-principal-de-los-protocolos-de-ventana-corrediza-en-la-capa-de-enlace}

Cada trama contiene un número de secuencia. En cualquier instante el
transmisor mantiene un grupo de números de secuencia de tramas enviadas
pero no reconocidas, que caen dentro de la ventana transmisora (cantidad
de tramas que el transmisor tiene permitido enviar sin esperar Ack) y el
receptor mantiene una ventana receptora correspondiente al número de
tramas que tiene permitido aceptar.

\subsubsection{\texorpdfstring{11. Un canal tiene una velocidad de 4kbps
y un retardo de propagación de 20ms. ¿Para qué rango de tamaño de trama
el método de ``Stop \& Wait'' (parada y espera) tiene una eficiencia de
utilización del enlace del
40\%?}{11. Un canal tiene una velocidad de 4kbps y un retardo de propagación de 20ms. ¿Para qué rango de tamaño de trama el método de Stop \& Wait (parada y espera) tiene una eficiencia de utilización del enlace del 40\%?}}\label{un-canal-tiene-una-velocidad-de-4kbps-y-un-retardo-de-propagaciuxf3n-de-20ms.-para-quuxe9-rango-de-tamauxf1o-de-trama-el-muxe9todo-de-stop-wait-parada-y-espera-tiene-una-eficiencia-de-utilizaciuxf3n-del-enlace-del-40}

\[ E = \dfrac{Ttr}{2Ttr + 2\tau} * 100 \]

\[ 0.4 = \dfrac{\dfrac{L_t}{Rb}}{2\dfrac{L_t}{Rb} + 2\tau} \]

\[ 0.4 * 2\dfrac{L_t}{4kb/s} + 40ms = \dfrac{L_t}{4kb/s} \]

\[ (0.8 \dfrac{L_t}{4kb/s} + 40ms) * 4kb/s = L_t \]

\[ 0.8 L_t + 40ms * \dfrac{1s}{1000ms} * 4kb/s = L_t \]

\[ 0.16kb = 0.2L_t \]

\[ 0.08kb = L_t \]

Para 80 bits de tamaño de trama la eficiencia es de 40\%

\subsubsection{12. ¿Qué ventajas tienen los protocolos de entubamiento y
repeticiones respecto a los protocolos de parada y
espera?}\label{quuxe9-ventajas-tienen-los-protocolos-de-entubamiento-y-repeticiones-respecto-a-los-protocolos-de-parada-y-espera}

Mejoran el aprovechamiento de ancho de banda del canal

\subsubsection{13. Interpretar el siguiente esquema de tranmisión de
tramas y completar las transferencias. Suponer entubamiento y Regresa
n.}\label{interpretar-el-siguiente-esquema-de-tranmisiuxf3n-de-tramas-y-completar-las-transferencias.-suponer-entubamiento-y-regresa-n.}

(En otra hoja)

\subsubsection{14. Un enlace de 2 Mbps consiste de dos subestaciones
separadas 50km, conectadas mediante cable coaxil, con 5 repetidoras que
poseen retardo propio de 4 µseg c/u. La longitud de las tramas es de
2000 bits y la velocidad de propagación en el cable de 200000km/s.
Calcular la eficiencia de utilización del enlace para los siguientes
casos:}\label{un-enlace-de-2-mbps-consiste-de-dos-subestaciones-separadas-50km-conectadas-mediante-cable-coaxil-con-5-repetidoras-que-poseen-retardo-propio-de-4-uxb5seg-cu.-la-longitud-de-las-tramas-es-de-2000-bits-y-la-velocidad-de-propagaciuxf3n-en-el-cable-de-200000kms.-calcular-la-eficiencia-de-utilizaciuxf3n-del-enlace-para-los-siguientes-casos}

\begin{itemize}
\itemsep1pt\parskip0pt\parsep0pt
\item
  S\&W
\end{itemize}

\[ E = \dfrac{Ttr}{2Ttr + 2\tau} * 100 \]

\[ Ttr = \dfrac{L_t}{Rb} \]

\[ Ttr = \dfrac{2kb}{2000kb/s} = 0.001s \]

\[ 200000km/s * \dfrac{1s}{1000000\mu s} = 0.2km/mu s \]

\[ \tau = \dfrac{50km}{0.2km/s} + 20\mu s = 270\mu s \]

\[ \tau = 270 \mu s * \dfrac{1s}{1000000\mu s} = 0.00027s \]

\[ E = \dfrac{0.001s}{0.002s + 0.00027s} * 100 = 44\% \]

\begin{itemize}
\itemsep1pt\parskip0pt\parsep0pt
\item
  Entubamiento con N=3
\end{itemize}

\[ E_{S\&W} * N = E_{N} \]

\[ E_{N=3} = 100\% \]

\begin{itemize}
\itemsep1pt\parskip0pt\parsep0pt
\item
  S\&W pero despreciando trama de acuse de recibo
\end{itemize}

\[ E = \dfrac{Ttr}{Ttr + 2\tau} * 100 \]

\[ E = \dfrac{0.001s}{0.001s + 0.00027s} * 100 = 44\% \]

\[ E = 78.7\% \]

\subsubsection{15. En un enlace a la luna de 1Mbps, se tiene un retardo
de ida de 1.28s, que corresponde a una distancia de 384400km a la
tierra. Suponiendo tramas de datos de 500 bytes, ¿cuál será la
eficiencia del sistema
si:}\label{en-un-enlace-a-la-luna-de-1mbps-se-tiene-un-retardo-de-ida-de-1.28s-que-corresponde-a-una-distancia-de-384400km-a-la-tierra.-suponiendo-tramas-de-datos-de-500-bytes-cuuxe1l-seruxe1-la-eficiencia-del-sistema-si}

\begin{itemize}
\itemsep1pt\parskip0pt\parsep0pt
\item
  Se desprecia el tiempo de acuse de recibo?
\end{itemize}

\[ E = \dfrac{Ttr}{2Ttr + 2\tau} * 100 \]

\[ Ttr = 0.004s \]

\[ E = \dfrac{0.004s}{0.004s + 2.56s} * 100 = 0.156\% \]

\begin{itemize}
\itemsep1pt\parskip0pt\parsep0pt
\item
  Se utiliza protocolo S\&W
\end{itemize}

\[ E = \dfrac{0.004s}{0.008s + 2.56s} * 100 = 0.156\% \]

\begin{itemize}
\itemsep1pt\parskip0pt\parsep0pt
\item
  Se emplea entubamiento con Vx = 127
\end{itemize}

\[ E_{S\&W} * N = E_{N} \]

\[ E_{N=27} = 19.78\% \]

\begin{itemize}
\itemsep1pt\parskip0pt\parsep0pt
\item
  ¿Cuál sería el tamaño mínimo de ventana de tranmisión para E = 100\%?
\end{itemize}

\[ N = \dfrac{100}{E_{S\&W}} \]

\[ N = 642 \]

\subsubsection{16. ¿Qué información contiene el formato de trama del
protocolo HDLC? ¿Es orientado a bits o a
carácter?}\label{quuxe9-informaciuxf3n-contiene-el-formato-de-trama-del-protocolo-hdlc-es-orientado-a-bits-o-a-caruxe1cter}

Orientado a bits, la organización de la información es la siguiente.

\begin{itemize}
\itemsep1pt\parskip0pt\parsep0pt
\item
  16 bits para relleno de bits
\item
  8 bits para dirección
\item
  8 bits para control
\item
  datos
\item
  16 bits de CRC
\end{itemize}

Hay 3 tipos de tramas, de información, de supervisión y no numeradas. Su
campo de control:

\textbar{} \textbar{} \textbar{}
::\textbar{}:------------------:\textbar{}:-----------:\textbar{}:------------------:
0 \textbar{} Secuencia (3 bits) \textbar{} S/F (1 bit) \textbar{}
Siguiente (3 bits) 10\textbar{} Tipo (2 bits) \textbar{} S/F (1 bit)
\textbar{} Siguiente (3 bits) 11\textbar{} Tipo (2 bits) \textbar{} S/F
(1 bit) \textbar{} Siguiente (3 bits)

S/F = Sondeo/Final, cuando una computadora sondea un grupo de
terminales. S invita a enviar datos y última trama termina con F.

Tipo = Tipo de trama de supervisión.

\subsubsection{17. ¿Qué protocolos emplea internet en la Capa de
Enlace?}\label{quuxe9-protocolos-emplea-internet-en-la-capa-de-enlace}

SLIP y PPP

\subsubsection{18. ¿Qué ventajas tiene PPP al emplear relleno de
carácteres para evitar indicadores accidentales en los datos respecto al
relleno de bits que emplea
HDLC?}\label{quuxe9-ventajas-tiene-ppp-al-emplear-relleno-de-caruxe1cteres-para-evitar-indicadores-accidentales-en-los-datos-respecto-al-relleno-de-bits-que-emplea-hdlc}

Garantiza una tranmisión más confiable de datos al facilitar la
sincronización.

    \begin{Verbatim}[commandchars=\\\{\}]
{\color{incolor}In [{\color{incolor}24}]:} \PY{l+m+mi}{100}\PY{o}{/}\PY{p}{(}\PY{l+m+mi}{100}\PY{o}{*}\PY{l+m+mf}{0.004}\PY{o}{/}\PY{p}{(}\PY{l+m+mf}{0.008}\PY{o}{+}\PY{l+m+mf}{2.56}\PY{p}{)}\PY{p}{)}
\end{Verbatim}

            \begin{Verbatim}[commandchars=\\\{\}]
{\color{outcolor}Out[{\color{outcolor}24}]:} 641.9999999999999
\end{Verbatim}
        
    \begin{Verbatim}[commandchars=\\\{\}]
{\color{incolor}In [{\color{incolor}16}]:} \PY{l+m+mf}{0.001}\PY{o}{/}\PY{n}{Out}\PY{p}{[}\PY{l+m+mi}{15}\PY{p}{]}
\end{Verbatim}

            \begin{Verbatim}[commandchars=\\\{\}]
{\color{outcolor}Out[{\color{outcolor}16}]:} 0.7874015748031495
\end{Verbatim}
        

    % Add a bibliography block to the postdoc
    
    
    
    \end{document}
